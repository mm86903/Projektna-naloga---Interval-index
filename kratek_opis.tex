\documentclass[12pt,a4paper]{article}
\usepackage[slovene]{babel}
\usepackage[utf8]{inputenc}
\usepackage[T1]{fontenc}
\usepackage{lmodern}
\usepackage{microtype}
\usepackage{amsmath,amssymb,amsthm}
\usepackage{geometry}


\DeclareMathOperator{\Int}{Int}
\theoremstyle{plain}
\newtheorem{definicija}{Definicija}[section]
\newtheorem{izrek}{Izrek}[section]



\title{ Interval index}
\author{Marija Mrvar in Jure Jerman \\ Projektna naloga}
\date{November 2025}



\begin{document}
\maketitle

\section{Uvod in definicije}


Naj bo $G = (V,E)$ končen, povezan in neusmerjen graf brez večkratnih povezav in zank.

\begin{definicija}
Za poljubni različni vozlišči $u, v \in V$ definiramo množico:
\[
I_G(u,v) = \bigl\{ w \in V \mid d_G(u,w) + d_G(w,v) = d_G(u,v) \bigr\}.
\]
Množica $I_G(u,v)$ vsebuje vsa vozlišča, ki ležijo na vsaj eni najkrajši $u$--$v$ poti v grafu $G$.
\end{definicija}


\begin{definicija}
Intervalni indeks grafa $G$ je definiran kot
\[
\Int(G) = \sum_{\substack{\{u,v\}\subset V}} \bigl( |I_G(u,v)| - 1 \bigr).
\]
\end{definicija}



Ekstremna primera sta:
\begin{itemize}
    \item Pot $P_n$
    \item Popoln graf $K_n$
\end{itemize}

\section{Cilji}

V okviru naloge se bomo osredotočili na dva glavna cilja:

\begin{enumerate}
    \item Dokazati, da med vsemi grafi z $n$ vozlišči intervalni indeks $\Int(G)$ maksimizira pot $P_n$.
    \item Med vsemi povezanimi kubičnimi (3-regularnimi) grafi na $n$ vozliščih poiskati:
    \begin{itemize}
        \item graf z minimalno vrednostjo $\Int(G)$,
        \item graf z maksimalno vrednostjo $\Int(G)$,
    \end{itemize}
    ter opisati strukturne lastnosti teh ekstremalnih grafov (premer, število najkrajših poti, itd.).
\end{enumerate}



\end{document}
