\documentclass[12pt, a4paper]{article}


\usepackage[utf8]{inputenc}
\usepackage[T1]{fontenc}

\usepackage[slovene]{babel}
\usepackage{lmodern}
\usepackage{amsmath}
\usepackage{amsfonts}
\usepackage{amsthm}
\usepackage{mathtools}
\usepackage{amsfonts}
\usepackage{enumitem}
\usepackage{ amssymb}
\usepackage{booktabs}


\usepackage{makeidx}

\usepackage{tikz} % za risanje
\usepackage{ulem}
\usepackage{enumitem}
\usepackage{graphicx} % za delo s slikami
\usepackage{hyperref}

\usepackage{placeins}
\usepackage{caption}


\theoremstyle{plain}
\newtheorem{definicija}{Definicija}[section]
\newtheorem{izrek}{Izrek}[section]

\DeclareMathOperator{\Int}{Int}



\begin{document}
\begin{titlepage}

    \raggedright
    {\large UNIVERZA V LJUBLJANI\\}
    {\large FAKULTETA ZA MATEMATIKO IN FIZIKO\\}
    \vspace{0.5cm}
    {\large Finančna matematika – 1. stopnja\\}
    
    \centering
    \vspace*{6cm} 
    {\large Marija Mrvar, Jure Jerman\\}
    \vspace{0.5cm}
    {\Huge\bfseries Interval index\\} 
    \vspace{1.0cm}
    {\large Projektna naloga\\}
    \vspace{0.5cm}
    {\large Mentorja: prof.\ dr.\ Riste Škrekovski,\\}
    {\large dr.\ Timotej Hrga\\}

    \vfill
    \raggedright
    {\large Ljubljana, 2025}

\end{titlepage}


\tableofcontents
% kazalo vsebine

\newpage

\section{Uvod}

V tem poročilu obravnavamo intervalni indeks $\Int(G)$, ki za graf $G$
meri, koliko vozlišč je vključenih na najkrajših poteh med pari vozlišč.
Višja kot je vrednost $\Int(G)$, bolj so sosednja vozlišča daleč narazen. \\

Zanimalo nas je dvoje:

\begin{enumerate}
    \item med vsemi povezanimi grafi na $n$ vozliščih testirati,
          ali ima pot $P_n$ največji intervalni indeks;
    \item med vsemi povezanimi kubičnimi (3-regularnimi) grafi na $n$ vozliščih
          poiskati graf z minimalnim in maksimalnim intervalnim indeksom ter
          opisati njune strukturne značilnosti.
\end{enumerate}

Za izračune smo uporabili SageMath.


\section{Osnovne definicije}
V vseh nadaljinih definicijah predpostavimo, da je $G = (V,E)$ končen, povezan in neusmerjen graf brez večkratnih povezav in zank.

\begin{definicija}
Za poljubni različni vozlišči $u, v \in V$ definiramo množico:
\[
I_G(u,v) = \bigl\{ w \in V \mid d_G(u,w) + d_G(w,v) = d_G(u,v) \bigr\}.
\]
Množica $I_G(u,v)$ vsebuje vsa vozlišča, ki ležijo na vsaj eni najkrajši $u$--$v$ poti v grafu $G$, vključno z vozliščema $u$ in $v$.
\end{definicija}


\begin{definicija}
Intervalni indeks grafa $G$ je definiran kot
\[
\Int(G) = \sum_{\substack{\{u,v\}\subset V}} \bigl( |I_G(u,v)| - 1 \bigr).
\]
\end{definicija}


\begin{definicija}
Graf $G=(V,E)$ je množica vozlišč $V=\{1,2,\dots,n\}$ in množica povezav $E$.
Pravimo, da je povezan, če med vsakima dvema vozliščema obstaja pot.
\end{definicija}

\begin{definicija}
Pot $P_n$ je preprost povezan graf z $n$ vozlišči in $n-1$ povezavami, pri čemer so vozlišča urejena v zaporedju, tako da je vsako vozlišče (razen prvega in zadnjega) povezano z natanko dvema sosedoma
\end{definicija}


\begin{definicija}
Graf je $k$-regularen, če ima vsako vozlišče stopnjo $k$.
Za $k=3$ govorimo o kubičnih grafih.
\end{definicija}


\section{Analiziranje intervalnega indeksa za pot $P_n$}

Najprej smo si ogledali intervalni indeks poti in pogledali vrednosti za vse $n < 100$. Očitno 
je bilo da indeks raste, kar je razvidno tudi v spodnnji sliki.

\begin{figure}[h]
    \centering
    \includegraphics[width=0.8\textwidth]{Posnetek zaslona 2025-12-10 104135.png}
    \caption{Rast intervalnega indeksa za pot}
\end{figure}

Primerjavo intervalnega indeksa $\Int(G)$ za pot $P_n$ z drugimi grafi pa smo analizirali na dva načina, in sicer, najprej smo uprabili 
ukaz geng in z njim definirali funkcijo \texttt{test\_path\_maximal}, s katerim smo primerjali indeks poti z indeksom vseh možnih grafov na vozliščih stopnje
$n \leq 9$, za večje $n$ pa nam s tem ukazom ni uspelo pokazati, saj je bilo grafov za prveriti preveč 
(poizkus preverjanja za $n = 10$ je namreč trajalo 1512 min in še ni uspelo). 

Tako smo za preverjanje večjih grafov ($n \geq 10$) definirali funkcijo \\ \texttt{test\_path\_vs\_random},
pri kateri smo uporabili ukaz graphs.RandomGNP, ki nam generira naključni graf po modelu Erdős–Rény
glede na podano verjetnost, ali ustvari rob ali ne. S tem smo si tako pomagali, da smo
testirali intervalni indeks poti in naključnjega grafa za $n = 100$ in v vseh primerih je imela pot tako kot
prej večji indeks v vseh poskusih. \\

\begin{figure}[h]
    \centering
    \includegraphics[width=0.8\textwidth]{Posnetek zaslona 2025-12-10 105834.png}
    \caption{Intervalni indeks poti in nakjučnega grafa za n = 100}
\end{figure}

\FloatBarrier %zato ker nam drugače slikico nekam cudno zamakne

\begin{figure}[h]
    \centering
    \includegraphics[width=0.8\textwidth]{slika0.png}
    \caption{Naključni graf z največjim interavlnim indkesom za $n = 100$}
\end{figure}

\FloatBarrier %zato ker nam drugače slikico nekam cudno zamakne


Ključni razlog, zakaj pot maksimira $\Int(G)$ med povezanimi grafi, je:
\begin{itemize}
    \item da je pot edini povezani graf, kjer ni nobene "bližnjice",
    \item vsak par sosednjih vozlišč je čim dlje narazen,
    \item dodajanje nove povezave v graf (npr.\ cikla)
          vedno povzroči nove najkraše poti.
\end{itemize}

Če povzamemo, lahko najpreprostejše rečemo, da poti maksimizirajo razdaljo med
vozlišči, ker je graf kar se da "raztegnjen" v eno smer in je tako med vozlišči kar se da velika razdalja.

\section{Uporaba metahevristike za iskanje ekstremov pri kubičnih grafih}

Ker je število povezanih kubičnih grafov na $n$ vozliščih za $n>10$ že 
zelo veliko, smo uporabili metahevristiko za iskanje dobrih (ne nujno optimalnih) rešitev v zelo velikih iskalnih prostorih.

Izbrali smo eno najpreprostejših metahevristik: 
\textbf{Hill-Climbing}. Algoritem deluje tako:

\begin{enumerate}
    \item Začnemo z naključnim povezanim kubičnim grafom $S$.
    \item V vsakem koraku naredimo majhno naključno spremembo ("tweak"): v našem primeru izberemo dva disjunktna roba in ju zamenjamo na enega od dveh možnih načinov (t.i. double edge swap<).
    \item Pomembno: sprememba mora ohraniti 3-regularnost grafa, zato vedno preverimo stopnje vseh vozlišč in sprejmemo samo veljavne zamenjave.
    \item Če ima novi graf večji (ali manjši) intervalni indeks kot trenutni, ga obdržimo in nadaljujemo iz njega.
    \item Postopek ponavljamo več tisočkrat, da dobimo čim boljše rezultate.
\end{enumerate}


Za majhne $n \leq 10$ smo kljub temu izvedli celoten pregled vseh kubičnih grafov z funkcijo \texttt{sistematicni\_kubicni},
Za večje $n > 10$  pa smo uporabili zgornjo metahevristiko z večjimi ponovitvami z funkcijo \\\texttt{metahevristika\_kubicni}.

\subsection{Uporabljene opazovane lastnosti grafov}
Za vsak ekstremni graf (z minimalnim oz. maksimalnim $\operatorname{Int}(G)$) smo izračunali:
\begin{itemize}
    \item premer grafa oz.\ diameter (najdaljša možna razdalja med dvema vozliščema),
    \item radij oz.\ radius (dolžina najdaljše najkrajše poti od v do katerega koli drugega vozlišča v grafu),
    \item obseg oz.\ girth (dolžina najmanjšega cikla ),
    \item bipartitnost oz.\ dvodelnost (razdelitev vozlišč v dve neodvisni, disjunktni množici),
    \item število robov,
    \item število vozlišč,
    \item hamiltonovost
\end{itemize}

Naredili smo analizo za vse kubične grafe za $n < 27$. Pri tem pa smo ugotovili naslednje:
\begin{itemize}
    \item Grafi z najmanjšim intervalnim indeksom so imeli majhen premer in velik obseg.
    Prav tako se je izkazalo tudi, da je bilo med njimi veliko grafov Hamiltonovih.
    \item Grafi z največjim indeksom imajo večji premer in strukturo, imajo pa za razliko od tistih 
    z manjšim indeksom, manjši obseg. Grafi so izgledali tudi precej simetrično in "podolgovati", nekoliko so spominjali na poti $P_n$.
    Zato je tudi logično, da zaradi večje razdlje med vozlišči, imajo večji intervalni indeks.
\end{itemize}

\pagebreak
\subsection{Grafični prikaz grafov}
\subsubsection{Grafa za $n = 10$}


\begin{figure}[!htbp]
    \centering
    
    \begin{minipage}[t]{0.25\textwidth}  % <-- [t] poravna po vrhu
        \centering
        \includegraphics[width=\textwidth]{slika1.png}
        \caption{Maksimlani indeks}
    \end{minipage}
    \hfill
    \begin{minipage}[t]{0.40\textwidth} 
        \includegraphics[width=\textwidth]{slika2.png}
        \caption{Minimalni indeks}
    \end{minipage}

\end{figure}



\subsubsection{Grafa za $n = 16$}

\begin{figure}[!htbp]
    \centering
    
    \begin{minipage}[t]{0.35\textwidth}  % <-- [t] poravna po vrhu
        \centering
        \includegraphics[width=\textwidth]{slika3.png}
        \caption{Maksimalni indkes}
    \end{minipage}
    \hfill
    \begin{minipage}[t]{0.35\textwidth} 
        \includegraphics[width=\textwidth]{slika4.png}
        \caption{Minimalni indeks}

    \end{minipage}

\end{figure}



%\FloatBarrier  
\pagebreak
\nocite{*}
\bibliographystyle{siam}
\bibliography{literatura}
\printindex

\end{document}

