\documentclass[12pt, a4paper]{article}


\usepackage[utf8]{inputenc}
\usepackage[T1]{fontenc}

\usepackage[slovene]{babel}
\usepackage{lmodern}
\usepackage{amsmath}
\usepackage{amsfonts}
\usepackage{amsthm}
\usepackage{mathtools}
\usepackage{amsfonts}
\usepackage{enumitem}
\usepackage{ amssymb}
\usepackage{booktabs}


\usepackage{makeidx}

\usepackage{tikz} % za risanje
\usepackage{ulem}
\usepackage{enumitem}
\usepackage{graphicx} % za delo s slikami
\usepackage{hyperref}


\theoremstyle{plain}
\newtheorem{definicija}{Definicija}[section]
\newtheorem{izrek}{Izrek}[section]

\DeclareMathOperator{\Int}{Int}




\begin{document}
\begin{titlepage}

    \raggedright
    {\large UNIVERZA V LJUBLJANI\\}
    {\large FAKULTETA ZA MATEMATIKO IN FIZIKO\\}
    \vspace{0.5cm}
    {\large Finančna matematika – 1. stopnja\\}
    
    \centering
    \vspace*{6cm} 
    {\large Marija Mrvar, Jure Jerman\\}
    \vspace{0.5cm}
    {\Huge\bfseries Interval index\\} 
    \vspace{1.0cm}
    {\large Projektna naloga\\}
    \vspace{0.5cm}
    {\large Mentorja: prof.\ dr.\ Riste Škrekovski,\\}
    {\large dr.\ Timotej Hrga\\}

    \vfill
    \raggedright
    {\large Ljubljana, 2025}

\end{titlepage}


\tableofcontents
% kazalo vsebine

\newpage

\section{Uvod}

V tem poročilu obravnavamo intervalni indeks $\Int(G)$, ki za graf $G$
meri, kako ``razpršene'' so oznake vozlišč, ki se pojavljajo na njegovih povezavah.
Višja kot je vrednost $\Int(G)$, bolj so sosednja vozlišča daleč narazen
po svoji oznaki.

Zanimalo nas je dvoje:

\begin{enumerate}
    \item med vsemi povezanimi grafi na $n$ vozliščih preveriti,
          ali ima pot $P_n$ največji intervalni indeks;
    \item med vsemi povezanimi kubičnimi (3-regularnimi) grafi na $n$ vozliščih
          poiskati graf z minimalnim in maksimalnim intervalnim indeksom ter
          opisati njune strukturne značilnosti.
\end{enumerate}

Za izračune smo uporabili SageMath.


\section{Osnovne definicije}
V vseh nadaljinih definicijah predpostavimo, da je $G = (V,E)$ končen, povezan in neusmerjen graf brez večkratnih povezav in zank.

\begin{definicija}
Za poljubni različni vozlišči $u, v \in V$ definiramo množico:
\[
I_G(u,v) = \bigl\{ w \in V \mid d_G(u,w) + d_G(w,v) = d_G(u,v) \bigr\}.
\]
Množica $I_G(u,v)$ vsebuje vsa vozlišča, ki ležijo na vsaj eni najkrajši $u$--$v$ poti v grafu $G$, vključno z vozliščema $u$ in $v$.
\end{definicija}


\begin{definicija}
Intervalni indeks grafa $G$ je definiran kot
\[
\Int(G) = \sum_{\substack{\{u,v\}\subset V}} \bigl( |I_G(u,v)| - 1 \bigr).
\]
\end{definicija}


\begin{definicija}
Graf $G=(V,E)$ je množica vozlišč $V=\{1,2,\dots,n\}$ in množica povezav $E$.
Pravimo, da je povezan, če med vsakima dvema vozliščema obstaja pot.
\end{definicija}

\begin{definicija}
Pot $P_n$ je preprost povezan graf z $n$ vozlišči in $n-1$ povezavami, pri čemer so vozlišča urejena v zaporedju, tako da je vsako vozlišče (razen prvega in zadnjega) povezano z natanko dvema sosedoma
\end{definicija}


\begin{definicija}
Graf je $k$-regularen, če ima vsako vozlišče stopnjo $k$.
Za $k=3$ govorimo o kubičnih grafih.
\end{definicija}


\section{Analiziranje intervalnega indeksa za pot $P_n$}

Najprej smo si ogledali intervalni indeks poti in pogledali vrednosti za vse $n < 100$. Očitno 
je bilo da indeks raste, kar je razvidno tudi v spodnnji sliki.

\begin{figure}[h]
    \centering
    \includegraphics[width=0.8\textwidth]{Posnetek zaslona 2025-12-10 104135.png}
    \caption{Rast intervalnega indeksa za pot}
\end{figure}

Primerjavo intervalnega indeksa $\Int(G)$ za pot $P_n$ z drugimi grafi pa smo analizirali na dva načina, in sicer, najprej smo uprabili 
ukaz geng in z njim definirali funkcijo \texttt{test\_path\_maximal}, s katerim smo primerjali indeks poti z indeksom vseh možnih grafov na vozliščih stopnje
$n \leq 9$, za večje $n$ pa nam s tem ukazom ni uspelo pokazati, saj je bilo grafov za prveriti preveč 
(poizkus preverjanja za $n = 10$ je namreč trajalo 1512 min in še ni uspelo). 

Tako smo za preverjanje večjih grafov ($n \geq 10$) definirali funkcijo \\ \texttt{test\_path\_vs\_random},
pri kateri smo uporabili ukaz graphs.RandomGNP, ki nam generira naključni graf po modelu Erdős–Rény
glede na podano verjetnost, ali ustvari rob ali ne. S tem smo si tako pomagali, da smo
testirali intervalni indeks poti in naključnjega grafa za $n = 100$ in v vseh primerih je imela pot tako kot
prej večji indeks v vseh poskusih. \\

\begin{figure}[h]
    \centering
    \includegraphics[width=0.8\textwidth]{Posnetek zaslona 2025-12-10 105834.png}
    \caption{Intervalni indeks poti in nakjučnega grafa za n = 100}
\end{figure}




Ključni razlog, zakaj pot maksimira $\Int(G)$ med povezanimi grafi, je:
\begin{itemize}
    \item da je pot edini povezani graf, kjer ni nobene "bližnjice",
    \item vsak par sosednjih vozlišč je v oznakah čim dlje narazen,
    \item dodajanje nove povezave v graf (npr. cikla)
          vedno povzroči nove najkraše poti.
\end{itemize}

Če povzamemo, lahko najpreprostejše rečemo, da poti maksimizirajo razdaljo med
vozlišči, ker je graf kar se da "raztegnjen" v eno smer.




\section{Zaključek}

Pokazali smo, da ima pot $P_n$ največji intervalni indeks med vsemi
povezanimi grafi iste velikosti.
Raziskali smo tudi kubične grafe in za vsako velikost določili grafa
z minimalnim in maksimalnim intervalnim indeksom.

Metoda se lahko razširi tudi na druge razrede grafov, npr.\ regularne grafe višjih stopenj,
drevesa ali naključne grafe.



\end{document}

